\documentclass[10pt,aspectratio=43,mathserif,table]{beamer} 
%设置为 Beamer 文档类型,设置字体为 10pt,长宽比为16:9,数学字体为 serif 风格
\batchmode

\usepackage{graphicx}
\usepackage{animate}
\usepackage{hyperref}

%导入一些用到的宏包
\usepackage{amsmath,bm,amsfonts,amssymb,enumerate,epsfig,bbm,calc,color,ifthen,capt-of,multimedia,hyperref}
\usepackage{xeCJK} %导入中文包
%\setCJKmainfont{SimHei} %字体采用黑体  Microsoft YaHei
\setCJKmainfont{WenQuanYi Micro Hei} %字体采用黑体  Microsoft YaHei

\usetheme{Berlin} %主题
\usecolortheme{sustech} %主题颜色

\usepackage[ruled,linesnumbered]{algorithm2e}

\usepackage{fancybox}
\usepackage{xcolor}
\usepackage{times}
\usepackage{listings}

\usepackage{booktabs}
\usepackage{colortbl}

\newcommand{\Console}{Console}
\lstset{ %
	backgroundcolor=\color{white},   % choose the background color
	basicstyle=\footnotesize\rmfamily,     % size of fonts used for the code
	columns=fullflexible,
	breaklines=true,                 % automatic line breaking only at whitespace
	captionpos=b,                    % sets the caption-position to bottom
	tabsize=4,
	commentstyle=\color{mygreen},    % comment style
	escapeinside={\%*}{*)},          % if you want to add LaTeX within your code
	keywordstyle=\color{blue},       % keyword style
	stringstyle=\color{mymauve}\ttfamily,     % string literal style
	numbers=left, 
	%	frame=single,
	rulesepcolor=\color{red!20!green!20!blue!20},
	% identifierstyle=\color{red},
	language=c
}

%\setsansfont{Microsoft YaHei}
%\setmainfont{Microsoft YaHei}

\setsansfont{WenQuanYi Micro Hei}
\setmainfont{WenQuanYi Micro Hei}

\definecolor{mygreen}{rgb}{0,0.6,0}
\definecolor{mymauve}{rgb}{0.58,0,0.82}
\definecolor{mygray}{gray}{.9}
\definecolor{mypink}{rgb}{.99,.91,.95}
\definecolor{mycyan}{cmyk}{.3,0,0,0}

%题目,作者,学校,日期
\title{NLP 实战课程}
\subtitle{\fontsize{9pt}{14pt}\textbf{}}
\author{丁贵金}
\institute{\fontsize{8pt}{14pt}草稿}
\date{\today}

%学校Logo
%\pgfdeclareimage[height=0.5cm]{sustech-logo}{sustech-logo.pdf}
%\logo{\pgfuseimage{sustech-logo}\hspace*{0.3cm}}

%%\AtBeginSection[]
%%{
%%	\begin{frame}<beamer>
%%	\frametitle{\textbf{目录}}
%%	\tableofcontents[currentsection]
%%\end{frame}
%%}
\beamerdefaultoverlayspecification{<+->}
% -----------------------------------------------------------------------------
\begin{document}
% -----------------------------------------------------------------------------

\frame{\titlepage}

\section[目录]{}   %目录
\begin{frame}{目录}
\tableofcontents
\end{frame}

% -----------------------------------------------------------------------------


\section{语音识别后处理} 

\begin{frame}{标点预测}
	语音识别的文本是没有标点的,要通过标点恢复/预测重新添加标点。
\begin{block}{}
	计划课时
	\begin{itemize}
		\item<0-> 2 小时
	\end{itemize}

	授课目的
	\begin{itemize}
		\item<0-> 深入理解BERT的FINETUNING/LSTM
	\end{itemize}

	预备知识
	\begin{itemize}
		\item<0-> BERT/TRANSFORMER/ATTENTION/EMBEDDING/LANGUAGE MODEL/LSTM/RNN
	\end{itemize}

	授课计划
	\begin{itemize}
		\item<0-> 原理介绍(BERT FINETUNING / LSTM)
		\item<0-> 代码讲解/演示训练/推理过程
		\item<0-> 探讨通过BERT进行其它文本任务和其他语音识别过程中NLP任务(提取/生成/热词)
	\end{itemize}

\end{block}
\end{frame}


\section{RASA ChatBot} 

\begin{frame}{ChatBot}
ChatBot 涉及的问题非常广,目前没有完美解决方案,通过 RASA 系统的实践,了解对话系统相关的任务和现状。
\begin{block}{}
	计划课时
	\begin{itemize}
		\item<0-> 4 小时
	\end{itemize}

	授课目的
	\begin{itemize}
		\item<0-> 通过实践 RASA 了解当下对话系统的基本构成,所要解决的问题。
	\end{itemize}

	预备知识
	\begin{itemize}
		\item<0-> Python
	\end{itemize}

	授课计划
	\begin{itemize}
		\item<0-> Demo RASA 工作流程
		\item<0-> 代码讲解
		\item<0-> 探讨改进的可能(通过集成BERT获取更好的词语表示等)
	\end{itemize}

\end{block}
\end{frame}


% -----------------------------------------------------------------------------
\end{document}
%文档结束
